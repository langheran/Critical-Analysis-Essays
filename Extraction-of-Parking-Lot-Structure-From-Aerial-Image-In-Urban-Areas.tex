\documentclass[]{article}
\usepackage{lmodern}
\usepackage{amssymb,amsmath}
\usepackage{ifxetex,ifluatex}
\usepackage{fixltx2e} % provides \textsubscript
\ifnum 0\ifxetex 1\fi\ifluatex 1\fi=0 % if pdftex
  \usepackage[T1]{fontenc}
  \usepackage[utf8]{inputenc}
\else % if luatex or xelatex
  \ifxetex
    \usepackage{mathspec}
  \else
    \usepackage{fontspec}
  \fi
  \defaultfontfeatures{Ligatures=TeX,Scale=MatchLowercase}
\fi
% use upquote if available, for straight quotes in verbatim environments
\IfFileExists{upquote.sty}{\usepackage{upquote}}{}
% use microtype if available
\IfFileExists{microtype.sty}{%
\usepackage{microtype}
\UseMicrotypeSet[protrusion]{basicmath} % disable protrusion for tt fonts
}{}
\usepackage[margin=1in]{geometry}
\usepackage{hyperref}
\PassOptionsToPackage{usenames,dvipsnames}{color} % color is loaded by hyperref
\hypersetup{unicode=true,
            pdftitle={Extraction of Parking Lot Structure From Aerial Image In Urban Areas},
            pdfauthor={Nisim Hurst},
            colorlinks=true,
            linkcolor=Maroon,
            citecolor=Blue,
            urlcolor=blue,
            breaklinks=true}
\urlstyle{same}  % don't use monospace font for urls
\usepackage[style=authoryear]{biblatex}

\addbibresource{Extraction-of-Parking-Lot-Structure-From-Aerial-Image-In-Urban-Areas/Extraction-of-Parking-Lot-Structure-From-Aerial-Image-In-Urban-Areas.bib}
\usepackage{longtable,booktabs}
\usepackage{graphicx,grffile}
\makeatletter
\def\maxwidth{\ifdim\Gin@nat@width>\linewidth\linewidth\else\Gin@nat@width\fi}
\def\maxheight{\ifdim\Gin@nat@height>\textheight\textheight\else\Gin@nat@height\fi}
\makeatother
% Scale images if necessary, so that they will not overflow the page
% margins by default, and it is still possible to overwrite the defaults
% using explicit options in \includegraphics[width, height, ...]{}
\setkeys{Gin}{width=\maxwidth,height=\maxheight,keepaspectratio}
\IfFileExists{parskip.sty}{%
\usepackage{parskip}
}{% else
\setlength{\parindent}{0pt}
\setlength{\parskip}{6pt plus 2pt minus 1pt}
}
\setlength{\emergencystretch}{3em}  % prevent overfull lines
\providecommand{\tightlist}{%
  \setlength{\itemsep}{0pt}\setlength{\parskip}{0pt}}
\setcounter{secnumdepth}{0}
% Redefines (sub)paragraphs to behave more like sections
\ifx\paragraph\undefined\else
\let\oldparagraph\paragraph
\renewcommand{\paragraph}[1]{\oldparagraph{#1}\mbox{}}
\fi
\ifx\subparagraph\undefined\else
\let\oldsubparagraph\subparagraph
\renewcommand{\subparagraph}[1]{\oldsubparagraph{#1}\mbox{}}
\fi

%%% Use protect on footnotes to avoid problems with footnotes in titles
\let\rmarkdownfootnote\footnote%
\def\footnote{\protect\rmarkdownfootnote}

%%% Change title format to be more compact
\usepackage{titling}

% Create subtitle command for use in maketitle
\newcommand{\subtitle}[1]{
  \posttitle{
    \begin{center}\large#1\end{center}
    }
}

\setlength{\droptitle}{-2em}
  \title{Extraction of Parking Lot Structure From Aerial Image In Urban Areas}
  \pretitle{\vspace{\droptitle}\centering\huge}
  \posttitle{\par}
  \author{\href{mailto:langheran@gmail.com}{Nisim Hurst}}
  \preauthor{\centering\large\emph}
  \postauthor{\par}
  \predate{\centering\large\emph}
  \postdate{\par}
  \date{Thursday 2 May 2019}

\usepackage{float}
\usepackage{graphicx}
\usepackage{subfig}
\usepackage{fancyhdr}
\pagestyle{fancy}
\usepackage{truncate}
\renewcommand{\subsectionmark}[1]{\markright{#1}{}}
\fancyhf{}
\lhead{\small\truncate{400pt}{\rightmark}}
\rhead{\small\hyperref[toc]{Table Of Contents}}
\rfoot{Page \thepage}
\usepackage{caption}
\usepackage{listings}
\usepackage{attachfile}
\makeatletter\renewcommand*{\fps@figure}{H}\makeatother
\usepackage{cleveref}

\usepackage{xcolor}
\definecolor{block-gray}{gray}{0.85}
\usepackage{environ}
\NewEnviron{quoteblock}
{\colorbox{block-gray}{
\parbox{\dimexpr\linewidth-2\fboxsep\relax}{
\small\addtolength{\leftskip}{10mm}
\addtolength{\rightskip}{10mm}
\BODY}}
}
\renewcommand{\quote}{\quoteblock}
\renewcommand{\endquote}{\endquoteblock}
\ifdef{\printbibliography}{
   \defbibheading{subsubbibliography}[\refname]{\subsubsection*{#1}}
   \let\oldprintbibliography\printbibliography
   \renewcommand{\printbibliography}[1]{
      \phantomsection
      \addcontentsline{toc}{section}{References}
      \oldprintbibliography[title={References},heading=subsubbibliography]
      }
}{

}

\begin{document}
\maketitle

\label{toc}

\hypertarget{extraction-of-parking-lot-structure-from-aerial-image-in-urban-areas}{%
\subsection{Extraction of Parking Lot Structure From Aerial Image In Urban Areas}\label{extraction-of-parking-lot-structure-from-aerial-image-in-urban-areas}}

The article was written by \autocite{koutaki2016extraction}. It was was cited \href{https://scholar.google.com/scholar?q=Extraction\%20of\%20Parking\%20Lot\%20Structure\%20From\%20Aerial\%20Image\%20In\%20Urban\%20Areas\%20koutaki\&hl=en\&as_sdt=0\&as_vis=1\&oi=scholart\&sa=X\&ved=0ahUKEwiN67qBocbWAhXrjVQKHcX0BYwQgQMIMDAA}{0} times according to Google Scholar. The task performed was detecting rectangular parking lot areas from aerial images. The metric for measuring performance is correctness., i.e.~the ratio between the correctly extracted parking lots and the total extracted parking lots. They also use a \% of detection metric, equal to the ratio between parking spots correctly extracted and the total real parking spots.

\hypertarget{hypothesis}{%
\subsubsection{Hypothesis}\label{hypothesis}}

Combining vehicle detection with parking spot detection will be useful to detect rectangular parking lot structures.

\hypertarget{evidence-and-results}{%
\subsubsection{Evidence and Results}\label{evidence-and-results}}

\hypertarget{dataset}{%
\paragraph{Dataset}\label{dataset}}

For vehicle detection 2520 positive image patches of 14x32 pixels were used. Likewise, 4800 negative patches were used. For parking space detection a single image patch of 15x28 pixels was used.

For parking lot detection, 4 images of 2000x2000 pixels were used. Zoom level is 20cm per pixel.

The images were used to train a Haar-like detector in conjunction with an AdaBoost ensemble classifier.

\hypertarget{results}{%
\paragraph{Results}\label{results}}

The parking lot rows were deduced using hierarchical grouping, starting with those parking spots that their centers are just one width of a parking spot apart.

The authors achieved 66.9\% completeness detecting vehicles, 30.2\% completeness for parking spot detection and correctness of 95\% in rectangular parking lot extraction. In the later case, a completeness of 100\% is assumed.

\hypertarget{contribution}{%
\subsubsection{Contribution}\label{contribution}}

First, the paper thoroughly defines the geometric structure and appearance model of the parking lot.

Second, high resolution elevation data is used to remove buildings. They combined a Digital Elevation Model and a Digital Surface Model with digital interpolation with the 4 urban zone images.

Third, a method for extracting those parking lots is proposed. This method is based on extracting both parking spaces and vehicle detection in parallel.

\hypertarget{weaknesses}{%
\subsubsection{Weaknesses}\label{weaknesses}}

No comparison with other methods is shown. All the parking lots are assumed to be rectangular. A single angle of 90 degrees between the parking spot and the parking row is assumed.

\hypertarget{future-work}{%
\subsubsection{Future Work}\label{future-work}}

For future work, the authors mention a shadow-resistant vehicle detection and to improve the single parking space extraction. Also, the authors mention that they wish to extend the algorithm to detect parking lots of arbitrary shape.

\href{gotosumatra:5\%7C32\%7C613\%7CG\%3A\%5CMy\%20Drive\%5CThesis\%5C13Feb\%5Fcomments\%5Creferences\%5CExtraction\%20of\%20Parking\%20Lot\%20Structure\%20From\%20Aerial\%20Image\%20In\%20Urban\%20Areas\%2Epdf}{Detection of vehicles}

\printbibliography


\end{document}
