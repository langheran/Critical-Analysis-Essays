\documentclass[]{article}
\usepackage{lmodern}
\usepackage{amssymb,amsmath}
\usepackage{ifxetex,ifluatex}
\usepackage{fixltx2e} % provides \textsubscript
\ifnum 0\ifxetex 1\fi\ifluatex 1\fi=0 % if pdftex
  \usepackage[T1]{fontenc}
  \usepackage[utf8]{inputenc}
\else % if luatex or xelatex
  \ifxetex
    \usepackage{mathspec}
  \else
    \usepackage{fontspec}
  \fi
  \defaultfontfeatures{Ligatures=TeX,Scale=MatchLowercase}
\fi
% use upquote if available, for straight quotes in verbatim environments
\IfFileExists{upquote.sty}{\usepackage{upquote}}{}
% use microtype if available
\IfFileExists{microtype.sty}{%
\usepackage{microtype}
\UseMicrotypeSet[protrusion]{basicmath} % disable protrusion for tt fonts
}{}
\usepackage[margin=1in]{geometry}
\usepackage{hyperref}
\PassOptionsToPackage{usenames,dvipsnames}{color} % color is loaded by hyperref
\hypersetup{unicode=true,
            pdftitle={A Vehicle Parking Detection Method Using Image Segmentation},
            pdfauthor={Nisim Hurst},
            colorlinks=true,
            linkcolor=Maroon,
            citecolor=Blue,
            urlcolor=blue,
            breaklinks=true}
\urlstyle{same}  % don't use monospace font for urls
\usepackage[style=authoryear]{biblatex}

\addbibresource{A-Vehicle-Parking-Detection-Method-Using-Image-Segmentation/A-Vehicle-Parking-Detection-Method-Using-Image-Segmentation.bib}
\usepackage{longtable,booktabs}
\usepackage{graphicx,grffile}
\makeatletter
\def\maxwidth{\ifdim\Gin@nat@width>\linewidth\linewidth\else\Gin@nat@width\fi}
\def\maxheight{\ifdim\Gin@nat@height>\textheight\textheight\else\Gin@nat@height\fi}
\makeatother
% Scale images if necessary, so that they will not overflow the page
% margins by default, and it is still possible to overwrite the defaults
% using explicit options in \includegraphics[width, height, ...]{}
\setkeys{Gin}{width=\maxwidth,height=\maxheight,keepaspectratio}
\IfFileExists{parskip.sty}{%
\usepackage{parskip}
}{% else
\setlength{\parindent}{0pt}
\setlength{\parskip}{6pt plus 2pt minus 1pt}
}
\setlength{\emergencystretch}{3em}  % prevent overfull lines
\providecommand{\tightlist}{%
  \setlength{\itemsep}{0pt}\setlength{\parskip}{0pt}}
\setcounter{secnumdepth}{0}
% Redefines (sub)paragraphs to behave more like sections
\ifx\paragraph\undefined\else
\let\oldparagraph\paragraph
\renewcommand{\paragraph}[1]{\oldparagraph{#1}\mbox{}}
\fi
\ifx\subparagraph\undefined\else
\let\oldsubparagraph\subparagraph
\renewcommand{\subparagraph}[1]{\oldsubparagraph{#1}\mbox{}}
\fi

%%% Use protect on footnotes to avoid problems with footnotes in titles
\let\rmarkdownfootnote\footnote%
\def\footnote{\protect\rmarkdownfootnote}

%%% Change title format to be more compact
\usepackage{titling}

% Create subtitle command for use in maketitle
\newcommand{\subtitle}[1]{
  \posttitle{
    \begin{center}\large#1\end{center}
    }
}

\setlength{\droptitle}{-2em}
  \title{A Vehicle Parking Detection Method Using Image Segmentation}
  \pretitle{\vspace{\droptitle}\centering\huge}
  \posttitle{\par}
  \author{\href{mailto:langheran@gmail.com}{Nisim Hurst}}
  \preauthor{\centering\large\emph}
  \postauthor{\par}
  \predate{\centering\large\emph}
  \postdate{\par}
  \date{Thursday 2 May 2019}

\usepackage{float}
\usepackage{graphicx}
\usepackage{subfig}
\usepackage{fancyhdr}
\pagestyle{fancy}
\usepackage{truncate}
\renewcommand{\subsectionmark}[1]{\markright{#1}{}}
\fancyhf{}
\lhead{\small\truncate{400pt}{\rightmark}}
\rhead{\small\hyperref[toc]{Table Of Contents}}
\rfoot{Page \thepage}
\usepackage{caption}
\usepackage{listings}
\usepackage{attachfile}
\makeatletter\renewcommand*{\fps@figure}{H}\makeatother
\usepackage{cleveref}

\usepackage{xcolor}
\definecolor{block-gray}{gray}{0.85}
\usepackage{environ}
\NewEnviron{quoteblock}
{\colorbox{block-gray}{
\parbox{\dimexpr\linewidth-2\fboxsep\relax}{
\small\addtolength{\leftskip}{10mm}
\addtolength{\rightskip}{10mm}
\BODY}}
}
\renewcommand{\quote}{\quoteblock}
\renewcommand{\endquote}{\endquoteblock}
\ifdef{\printbibliography}{
   \defbibheading{subsubbibliography}[\refname]{\subsubsection*{#1}}
   \let\oldprintbibliography\printbibliography
   \renewcommand{\printbibliography}[1]{
      \phantomsection
      \addcontentsline{toc}{section}{References}
      \oldprintbibliography[title={References},heading=subsubbibliography]
      }
}{

}

\begin{document}
\maketitle

\label{toc}

\hypertarget{a-vehicle-parking-detection-method-using-image-segmentation}{%
\subsection{A Vehicle Parking Detection Method Using Image Segmentation}\label{a-vehicle-parking-detection-method-using-image-segmentation}}

\hypertarget{abstract}{%
\subsubsection{Abstract}\label{abstract}}

The article was written by \autocite{Yamada_2001} on 2001. It was cited \href{https://scholar.google.com/scholar?cluster=46549803554427231\&hl=en\&as_sdt=2005\&sciodt=0,5}{41} times according to Google Scholar. The task performed was occupancy level detection using true positive rate over pre-annotated parking spots (stalls).

\hypertarget{hypothesis}{%
\subsubsection{Hypothesis}\label{hypothesis}}

Using features only from gray segments extracted from color histograms can achieve high true positive rate occupancy level detection.

\hypertarget{evidence-and-results}{%
\subsubsection{Evidence and Results}\label{evidence-and-results}}

\hypertarget{dataset}{%
\paragraph{Dataset}\label{dataset}}

The training dataset consist of 129 images extracted during 3 days in intervals of 20 minutes from an outdoor camera. From those images a total of 5182 annotated cells were extracted. A threshold for the gray level was determined from those cells. The cells are marked as occupied or vacant.

The test dataset consist of 165 images extracted during 4 days at intervals of 20 minutes. 6733 cells then were used to evaluate the detection rate (true positive rate) and misdetection (false positive rate). Nighttime images from 2 days also were used.

\hypertarget{evidence}{%
\paragraph{Evidence}\label{evidence}}

The authors variate the resolution per cell, night vs day and \{rain, cloudy, clear\} weather conditions. This results are presented in tabular form with a fixed gray threshold level or the optimal threshold level per day.

All the results are presented using detection rate that is assumed to be the same as true positive rate.

\hypertarget{contribution}{%
\subsubsection{Contribution}\label{contribution}}

The authors proved occupancy detection can be done with 98.7\% TPR by training a threshold over gray level.
Given that shape features are completely ignored, the algorithm is robust to different car shapes and weather conditions.

\hypertarget{weaknesses}{%
\subsubsection{Weaknesses}\label{weaknesses}}

The dataset consist solely of images from sunrise to sunset and nighttime, so the algorithm is not capable to learn a different set of gray thresholds for twilight conditions.

Likewise, a single TV camera was mounted on a neighboring building. Thus, there is still the need to test if the algorithm generalize well under different conditions, specially different altitudes and indoor environments.

\hypertarget{future-work}{%
\subsubsection{Future Work}\label{future-work}}

Yamada and Mizuno propose to extend the algorithm for nighttime detection by a suitable definition of the metric used.

\printbibliography


\end{document}
