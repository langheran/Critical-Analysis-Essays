\documentclass[]{article}
\usepackage{lmodern}
\usepackage{amssymb,amsmath}
\usepackage{ifxetex,ifluatex}
\usepackage{fixltx2e} % provides \textsubscript
\ifnum 0\ifxetex 1\fi\ifluatex 1\fi=0 % if pdftex
  \usepackage[T1]{fontenc}
  \usepackage[utf8]{inputenc}
\else % if luatex or xelatex
  \ifxetex
    \usepackage{mathspec}
  \else
    \usepackage{fontspec}
  \fi
  \defaultfontfeatures{Ligatures=TeX,Scale=MatchLowercase}
\fi
% use upquote if available, for straight quotes in verbatim environments
\IfFileExists{upquote.sty}{\usepackage{upquote}}{}
% use microtype if available
\IfFileExists{microtype.sty}{%
\usepackage{microtype}
\UseMicrotypeSet[protrusion]{basicmath} % disable protrusion for tt fonts
}{}
\usepackage[margin=1in]{geometry}
\usepackage{hyperref}
\PassOptionsToPackage{usenames,dvipsnames}{color} % color is loaded by hyperref
\hypersetup{unicode=true,
            pdftitle={A Method for Camera Vision Based Parking Spot Detection},
            pdfauthor={Nisim Hurst},
            colorlinks=true,
            linkcolor=Maroon,
            citecolor=Blue,
            urlcolor=blue,
            breaklinks=true}
\urlstyle{same}  % don't use monospace font for urls
\usepackage[style=authoryear]{biblatex}

\addbibresource{A-Method-for-Camera-Vision-Based-Parking-Spot-Detection/A-Method-for-Camera-Vision-Based-Parking-Spot-Detection.bib}
\usepackage{longtable,booktabs}
\usepackage{graphicx,grffile}
\makeatletter
\def\maxwidth{\ifdim\Gin@nat@width>\linewidth\linewidth\else\Gin@nat@width\fi}
\def\maxheight{\ifdim\Gin@nat@height>\textheight\textheight\else\Gin@nat@height\fi}
\makeatother
% Scale images if necessary, so that they will not overflow the page
% margins by default, and it is still possible to overwrite the defaults
% using explicit options in \includegraphics[width, height, ...]{}
\setkeys{Gin}{width=\maxwidth,height=\maxheight,keepaspectratio}
\IfFileExists{parskip.sty}{%
\usepackage{parskip}
}{% else
\setlength{\parindent}{0pt}
\setlength{\parskip}{6pt plus 2pt minus 1pt}
}
\setlength{\emergencystretch}{3em}  % prevent overfull lines
\providecommand{\tightlist}{%
  \setlength{\itemsep}{0pt}\setlength{\parskip}{0pt}}
\setcounter{secnumdepth}{0}
% Redefines (sub)paragraphs to behave more like sections
\ifx\paragraph\undefined\else
\let\oldparagraph\paragraph
\renewcommand{\paragraph}[1]{\oldparagraph{#1}\mbox{}}
\fi
\ifx\subparagraph\undefined\else
\let\oldsubparagraph\subparagraph
\renewcommand{\subparagraph}[1]{\oldsubparagraph{#1}\mbox{}}
\fi

%%% Use protect on footnotes to avoid problems with footnotes in titles
\let\rmarkdownfootnote\footnote%
\def\footnote{\protect\rmarkdownfootnote}

%%% Change title format to be more compact
\usepackage{titling}

% Create subtitle command for use in maketitle
\newcommand{\subtitle}[1]{
  \posttitle{
    \begin{center}\large#1\end{center}
    }
}

\setlength{\droptitle}{-2em}
  \title{A Method for Camera Vision Based Parking Spot Detection}
  \pretitle{\vspace{\droptitle}\centering\huge}
  \posttitle{\par}
  \author{\href{mailto:langheran@gmail.com}{Nisim Hurst}}
  \preauthor{\centering\large\emph}
  \postauthor{\par}
  \predate{\centering\large\emph}
  \postdate{\par}
  \date{Thursday 2 May 2019}

\usepackage{float}
\usepackage{graphicx}
\usepackage{subfig}
\usepackage{fancyhdr}
\pagestyle{fancy}
\usepackage{truncate}
\renewcommand{\subsectionmark}[1]{\markright{#1}{}}
\fancyhf{}
\lhead{\small\truncate{400pt}{\rightmark}}
\rhead{\small\hyperref[toc]{Table Of Contents}}
\rfoot{Page \thepage}
\usepackage{caption}
\usepackage{listings}
\usepackage{attachfile}
\makeatletter\renewcommand*{\fps@figure}{H}\makeatother
\usepackage{cleveref}

\usepackage{xcolor}
\definecolor{block-gray}{gray}{0.85}
\usepackage{environ}
\NewEnviron{quoteblock}
{\colorbox{block-gray}{
\parbox{\dimexpr\linewidth-2\fboxsep\relax}{
\small\addtolength{\leftskip}{10mm}
\addtolength{\rightskip}{10mm}
\BODY}}
}
\renewcommand{\quote}{\quoteblock}
\renewcommand{\endquote}{\endquoteblock}
\ifdef{\printbibliography}{
   \defbibheading{subsubbibliography}[\refname]{\subsubsection*{#1}}
   \let\oldprintbibliography\printbibliography
   \renewcommand{\printbibliography}[1]{
      \phantomsection
      \addcontentsline{toc}{section}{References}
      \oldprintbibliography[title={References},heading=subsubbibliography]
      }
}{

}

\begin{document}
\maketitle

\label{toc}

\hypertarget{a-method-for-camera-vision-based-parking-spot-detection}{%
\subsection{A Method for Camera Vision Based Parking Spot Detection}\label{a-method-for-camera-vision-based-parking-spot-detection}}

The article was written by \autocite{Weis_2006}. It was was cited \href{https://scholar.google.com/scholar?cites=14395139259540139730\&as_sdt=2005\&sciodt=0,5\&hl=en}{3} times according to Google Scholar. The task performed was detecting parking spot behind a vehicle from its rear camera.

\hypertarget{hypothesis}{%
\subsubsection{Hypothesis}\label{hypothesis}}

By using color for edge detection using a Kirsch filter it is possible to detect available parking spots.

\hypertarget{evidence-and-results}{%
\subsubsection{Evidence and Results}\label{evidence-and-results}}

Four images show parking spot detection. The first shows a parking spot occluded by a car being detected. The second image shows a free parking spot being detected. The third and fourth images show rejections.

\hypertarget{contribution}{%
\subsubsection{Contribution}\label{contribution}}

A first contribution is a thorough problem characterization of detecting a parking spot from a frontal camera. Two different parking spots poses are considered: perpendicular and parallel.

The whole method is based in detecting the parking lines and then apply fixed rules to filter them over a semantic analysis.

First, the authors pre-process the image to filter out colors and gray levels not belonging to classical parking lines. Then, the authors use a Kirsh filter to vectorize the lines. Finally, semantical filtering is applied.

\hypertarget{weaknesses}{%
\subsubsection{Weaknesses}\label{weaknesses}}

No quantitative results are show, just a red overlay over the parking lines and a green frame over the surmised parking spot.

A range on the size of a parking spot, i.e.~its height and width, is assumed.

\hypertarget{future-work}{%
\subsubsection{Future Work}\label{future-work}}

Apply learning of some kind rather than assume fixed ranges.

\printbibliography


\end{document}
