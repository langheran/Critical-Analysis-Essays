\documentclass[17pt,]{extarticle}
\usepackage{lmodern}
\usepackage{setspace}
\setstretch{1.5}
\usepackage{amssymb,amsmath}
\usepackage{ifxetex,ifluatex}
\usepackage{fixltx2e} % provides \textsubscript
\ifnum 0\ifxetex 1\fi\ifluatex 1\fi=0 % if pdftex
  \usepackage[T1]{fontenc}
  \usepackage[utf8]{inputenc}
\else % if luatex or xelatex
  \ifxetex
    \usepackage{mathspec}
  \else
    \usepackage{fontspec}
  \fi
  \defaultfontfeatures{Ligatures=TeX,Scale=MatchLowercase}
\fi
% use upquote if available, for straight quotes in verbatim environments
\IfFileExists{upquote.sty}{\usepackage{upquote}}{}
% use microtype if available
\IfFileExists{microtype.sty}{%
\usepackage{microtype}
\UseMicrotypeSet[protrusion]{basicmath} % disable protrusion for tt fonts
}{}
\usepackage[left=1cm, right=1cm, top=2.5cm, bottom=2.5cm]{geometry}
\usepackage{hyperref}
\PassOptionsToPackage{usenames,dvipsnames}{color} % color is loaded by hyperref
\hypersetup{unicode=true,
            pdftitle={Clustering Millions of Faces By Identity},
            pdfauthor={Nisim Hurst},
            colorlinks=true,
            linkcolor=Maroon,
            citecolor=Blue,
            urlcolor=blue,
            breaklinks=true}
\urlstyle{same}  % don't use monospace font for urls
\usepackage[style=authoryear]{biblatex}

\addbibresource{Clustering-Millions-of-Faces-By-Identity/Clustering-Millions-of-Faces-By-Identity.bib}
\usepackage{longtable,booktabs}
\usepackage{graphicx,grffile}
\makeatletter
\def\maxwidth{\ifdim\Gin@nat@width>\linewidth\linewidth\else\Gin@nat@width\fi}
\def\maxheight{\ifdim\Gin@nat@height>\textheight\textheight\else\Gin@nat@height\fi}
\makeatother
% Scale images if necessary, so that they will not overflow the page
% margins by default, and it is still possible to overwrite the defaults
% using explicit options in \includegraphics[width, height, ...]{}
\setkeys{Gin}{width=\maxwidth,height=\maxheight,keepaspectratio}
\IfFileExists{parskip.sty}{%
\usepackage{parskip}
}{% else
\setlength{\parindent}{0pt}
\setlength{\parskip}{6pt plus 2pt minus 1pt}
}
\setlength{\emergencystretch}{3em}  % prevent overfull lines
\providecommand{\tightlist}{%
  \setlength{\itemsep}{0pt}\setlength{\parskip}{0pt}}
\setcounter{secnumdepth}{0}
% Redefines (sub)paragraphs to behave more like sections
\ifx\paragraph\undefined\else
\let\oldparagraph\paragraph
\renewcommand{\paragraph}[1]{\oldparagraph{#1}\mbox{}}
\fi
\ifx\subparagraph\undefined\else
\let\oldsubparagraph\subparagraph
\renewcommand{\subparagraph}[1]{\oldsubparagraph{#1}\mbox{}}
\fi

%%% Use protect on footnotes to avoid problems with footnotes in titles
\let\rmarkdownfootnote\footnote%
\def\footnote{\protect\rmarkdownfootnote}

%%% Change title format to be more compact
\usepackage{titling}

% Create subtitle command for use in maketitle
\newcommand{\subtitle}[1]{
  \posttitle{
    \begin{center}\large#1\end{center}
    }
}

\setlength{\droptitle}{-2em}
  \title{Clustering Millions of Faces By Identity}
  \pretitle{\vspace{\droptitle}\centering\huge}
  \posttitle{\par}
  \author{\href{mailto:langheran@gmail.com}{Nisim Hurst}}
  \preauthor{\centering\large\emph}
  \postauthor{\par}
  \predate{\centering\large\emph}
  \postdate{\par}
  \date{Tuesday 14 May 2019}

\usepackage{float}
\usepackage{graphicx}
\usepackage{subfig}
\usepackage{fancyhdr}
\pagestyle{fancy}
\usepackage{truncate}
\renewcommand{\subsectionmark}[1]{\markright{#1}{}}
\fancyhf{}
\lhead{\small\truncate{400pt}{\rightmark}}
\rhead{\small\hyperref[toc]{Table Of Contents}}
\rfoot{Page \thepage}
\usepackage{caption}
\usepackage{listings}
\usepackage{attachfile}
\makeatletter\renewcommand*{\fps@figure}{H}\makeatother
\usepackage{cleveref}

\usepackage{xcolor}
\definecolor{block-gray}{gray}{0.85}
\usepackage{environ}
\NewEnviron{quoteblock}
{\colorbox{block-gray}{
\parbox{\dimexpr\linewidth-2\fboxsep\relax}{
\small\addtolength{\leftskip}{10mm}
\addtolength{\rightskip}{10mm}
\BODY}}
}
\renewcommand{\quote}{\quoteblock}
\renewcommand{\endquote}{\endquoteblock}
\ifdef{\printbibliography}{
   \defbibheading{subsubbibliography}[\refname]{\subsubsection*{#1}}
   \let\oldprintbibliography\printbibliography
   \renewcommand{\printbibliography}[1]{
      \phantomsection
      \addcontentsline{toc}{section}{References}
      \oldprintbibliography[title={References},heading=subsubbibliography]
      }
}{

}

\begin{document}
\maketitle

\label{toc}

\hypertarget{clustering-millions-of-faces-by-identity}{%
\subsection{Clustering Millions of Faces by Identity}\label{clustering-millions-of-faces-by-identity}}

The article was written by \autocite{otto2018}. It was was cited \href{https://scholar.google.com/scholar?cites=9743611198042490448\&as_sdt=2005\&sciodt=0,5\&hl=en}{44} times according to Google Scholar. The task performed was face clustering. They used the Pairwise F-measure metric over clusters with distractor images. They also developed their own metric for measuring internal cluster quality using just the k-top nearest neighbors.

\hypertarget{hypothesis}{%
\subsubsection{Hypothesis}\label{hypothesis}}

Deep features clustered using only the top-k nearest neighbors in an approximate rank-order clustering will produce a more scalable and a more accurate face clustering algorithm. This algorithm will be able to overcome the presence of millions distractor images and class imbalance.

The network architecture to produce a 320D feature vector was VGG16 proposed by \autocite{Simonyan2014}. The rank-order clustering algorithm is based on \autocite{zhu2011}. Their k-d tree implementation for calculating just the 200-top nearest neighbors is based on \autocite{muja2014}.

\hypertarget{evidence-and-results}{%
\subsubsection{Evidence and Results}\label{evidence-and-results}}

Evidence is presented first over a small dataset and the over an augmented version of the datasets with millions of distractor images.

\hypertarget{dataset}{%
\paragraph{Dataset}\label{dataset}}

The feature extractor was trained with the CASIA-webface. LFW, YTF were used for cluster evaluation, the former over static images and the latter over videos. Webfaces was used to augment the LFW. Here is a brief description of each:

\small

\begin{longtable}[]{@{}lllll@{}}
\caption{\label{tab:table1} Main characteristics of the four datasets that were used to test the improved CW.}\tabularnewline
\toprule
\begin{minipage}[b]{0.09\columnwidth}\raggedright
\strut
\end{minipage} & \begin{minipage}[b]{0.24\columnwidth}\raggedright
\# Instances\strut
\end{minipage} & \begin{minipage}[b]{0.14\columnwidth}\raggedright
Resolution\strut
\end{minipage} & \begin{minipage}[b]{0.19\columnwidth}\raggedright
Scenery\strut
\end{minipage} & \begin{minipage}[b]{0.20\columnwidth}\raggedright
Author\strut
\end{minipage}\tabularnewline
\midrule
\endfirsthead
\toprule
\begin{minipage}[b]{0.09\columnwidth}\raggedright
\strut
\end{minipage} & \begin{minipage}[b]{0.24\columnwidth}\raggedright
\# Instances\strut
\end{minipage} & \begin{minipage}[b]{0.14\columnwidth}\raggedright
Resolution\strut
\end{minipage} & \begin{minipage}[b]{0.19\columnwidth}\raggedright
Scenery\strut
\end{minipage} & \begin{minipage}[b]{0.20\columnwidth}\raggedright
Author\strut
\end{minipage}\tabularnewline
\midrule
\endhead
\begin{minipage}[t]{0.09\columnwidth}\raggedright
LFW\strut
\end{minipage} & \begin{minipage}[t]{0.24\columnwidth}\raggedright
13233 images of 5749. Only 1680 subjects have two or more photos.\strut
\end{minipage} & \begin{minipage}[t]{0.14\columnwidth}\raggedright
??, variable head angle\strut
\end{minipage} & \begin{minipage}[t]{0.19\columnwidth}\raggedright
Color, different Poses and Backgrounds.\strut
\end{minipage} & \begin{minipage}[t]{0.20\columnwidth}\raggedright
\autocite{huang2008}\strut
\end{minipage}\tabularnewline
\begin{minipage}[t]{0.09\columnwidth}\raggedright
YTF\strut
\end{minipage} & \begin{minipage}[t]{0.24\columnwidth}\raggedright
3425 videos of 1595 subjects.\strut
\end{minipage} & \begin{minipage}[t]{0.14\columnwidth}\raggedright
100x100, variable enclosing area\strut
\end{minipage} & \begin{minipage}[t]{0.19\columnwidth}\raggedright
Color, different Poses and Backgrounds.\strut
\end{minipage} & \begin{minipage}[t]{0.20\columnwidth}\raggedright
\autocite{wolf2011}\strut
\end{minipage}\tabularnewline
\begin{minipage}[t]{0.09\columnwidth}\raggedright
Webfaces\strut
\end{minipage} & \begin{minipage}[t]{0.24\columnwidth}\raggedright
123,654,141 distractor images.\strut
\end{minipage} & \begin{minipage}[t]{0.14\columnwidth}\raggedright
N/A\strut
\end{minipage} & \begin{minipage}[t]{0.19\columnwidth}\raggedright
N/A\strut
\end{minipage} & \begin{minipage}[t]{0.20\columnwidth}\raggedright
\autocite{otto2018}\strut
\end{minipage}\tabularnewline
\begin{minipage}[t]{0.09\columnwidth}\raggedright
CASIA-webface\strut
\end{minipage} & \begin{minipage}[t]{0.24\columnwidth}\raggedright
494,414 images of 10,575 subjects.\strut
\end{minipage} & \begin{minipage}[t]{0.14\columnwidth}\raggedright
120x165\strut
\end{minipage} & \begin{minipage}[t]{0.19\columnwidth}\raggedright
Color, different Poses and Backgrounds.\strut
\end{minipage} & \begin{minipage}[t]{0.20\columnwidth}\raggedright
\autocite{yi2014}\strut
\end{minipage}\tabularnewline
\bottomrule
\end{longtable}

\normalsize

\hypertarget{results}{%
\paragraph{Results}\label{results}}

First, the authors present Pairwise F-measure evaluated in the LFW dataset without any distractor images. The algorithm obtained the highest F-Measure and lowest run-time. The algorithm is proficient at handling class unbalance.

Second, the authors show performance on the augmented LFW dataset and the decay rate of each algorithm under these conditions. The proposed algorithm shows the highest resiliency compared to the decays ensued in the other algorithm.

Having benchmarked the algorithm in contrast to the other methods, the authors estimate internal performance under increasing levels of distractor images and search spaces.

Also, given that having a high number of similar frames on each video can affect grouping identities between videos, the authors present the results of the algorithm using a sample of 3 frames per video in contrast to the results obtained over all the frames.

Finally, the authors presents

\hypertarget{contribution}{%
\subsubsection{Contribution}\label{contribution}}

Firstly, the authors improved the Rank-Order clustering algorithm proposed by \autocite{zhu2011}. The original Rank-Order has the disadvantage that it requires \(O(n^2)\). The authors propose to use the FLANN library implementation of the randomized k-d tree algorithm to compute the list of top-k nearest neighbors \autocite{hartley2008}. Just one iteration is used. This approximate version had better performance compared to the exact rank-order and was faster than all the methods tested.

Secondly, the authors improved the internal quality metric of Modularization quality (MQ) \autocite{mancoridis1998} by just counting shared neighbors in the top-k nearest neighbors list. Cluster's external quality was obviated.

Thirdly, the authors provide an augmented dataset as a matter of baseline to assess the accuracy of the algorithm under the effect of distractor images that are out of the face clusters.

\hypertarget{weaknesses}{%
\subsubsection{Weaknesses}\label{weaknesses}}

The method using the dataset with the full 123M distractor images produces a representation that needs to be distributed in chunks across servers, each one process about a million image instances. However, the authors don't provide an efficient algorithm for merging the results nor prove that the algorithm is unaffected in single-thread environments.

Also, the method is dependent of a \(k\) that depends on the number of instances, but the authors don't specify how \(k\) should be modified. They tested with different

\hypertarget{future-work}{%
\subsubsection{Future Work}\label{future-work}}

Otto et al.~mentions that the dimensional vector representation could be improved through a better deep model architect that perform better on profile/side faces.

It would be beneficial to enforce pairwise constraints like must-link and cannot-link. Also the authors

\printbibliography


\end{document}
