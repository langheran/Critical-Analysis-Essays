\documentclass[]{article}
\usepackage{lmodern}
\usepackage{amssymb,amsmath}
\usepackage{ifxetex,ifluatex}
\usepackage{fixltx2e} % provides \textsubscript
\ifnum 0\ifxetex 1\fi\ifluatex 1\fi=0 % if pdftex
  \usepackage[T1]{fontenc}
  \usepackage[utf8]{inputenc}
\else % if luatex or xelatex
  \ifxetex
    \usepackage{mathspec}
  \else
    \usepackage{fontspec}
  \fi
  \defaultfontfeatures{Ligatures=TeX,Scale=MatchLowercase}
\fi
% use upquote if available, for straight quotes in verbatim environments
\IfFileExists{upquote.sty}{\usepackage{upquote}}{}
% use microtype if available
\IfFileExists{microtype.sty}{%
\usepackage{microtype}
\UseMicrotypeSet[protrusion]{basicmath} % disable protrusion for tt fonts
}{}
\usepackage[margin=1in]{geometry}
\usepackage{hyperref}
\PassOptionsToPackage{usenames,dvipsnames}{color} % color is loaded by hyperref
\hypersetup{unicode=true,
            pdftitle={A method of Parking space detection based on image segmentation and LBP},
            pdfauthor={Nisim Hurst},
            colorlinks=true,
            linkcolor=Maroon,
            citecolor=Blue,
            urlcolor=blue,
            breaklinks=true}
\urlstyle{same}  % don't use monospace font for urls
\usepackage[style=authoryear]{biblatex}

\addbibresource{A-method-of-Parking-space-detection-based-on-image-segmentation-and-LBP/A-method-of-Parking-space-detection-based-on-image-segmentation-and-LBP.bib}
\usepackage{longtable,booktabs}
\usepackage{graphicx,grffile}
\makeatletter
\def\maxwidth{\ifdim\Gin@nat@width>\linewidth\linewidth\else\Gin@nat@width\fi}
\def\maxheight{\ifdim\Gin@nat@height>\textheight\textheight\else\Gin@nat@height\fi}
\makeatother
% Scale images if necessary, so that they will not overflow the page
% margins by default, and it is still possible to overwrite the defaults
% using explicit options in \includegraphics[width, height, ...]{}
\setkeys{Gin}{width=\maxwidth,height=\maxheight,keepaspectratio}
\IfFileExists{parskip.sty}{%
\usepackage{parskip}
}{% else
\setlength{\parindent}{0pt}
\setlength{\parskip}{6pt plus 2pt minus 1pt}
}
\setlength{\emergencystretch}{3em}  % prevent overfull lines
\providecommand{\tightlist}{%
  \setlength{\itemsep}{0pt}\setlength{\parskip}{0pt}}
\setcounter{secnumdepth}{0}
% Redefines (sub)paragraphs to behave more like sections
\ifx\paragraph\undefined\else
\let\oldparagraph\paragraph
\renewcommand{\paragraph}[1]{\oldparagraph{#1}\mbox{}}
\fi
\ifx\subparagraph\undefined\else
\let\oldsubparagraph\subparagraph
\renewcommand{\subparagraph}[1]{\oldsubparagraph{#1}\mbox{}}
\fi

%%% Use protect on footnotes to avoid problems with footnotes in titles
\let\rmarkdownfootnote\footnote%
\def\footnote{\protect\rmarkdownfootnote}

%%% Change title format to be more compact
\usepackage{titling}

% Create subtitle command for use in maketitle
\newcommand{\subtitle}[1]{
  \posttitle{
    \begin{center}\large#1\end{center}
    }
}

\setlength{\droptitle}{-2em}
  \title{A method of Parking space detection based on image segmentation and LBP}
  \pretitle{\vspace{\droptitle}\centering\huge}
  \posttitle{\par}
  \author{\href{mailto:langheran@gmail.com}{Nisim Hurst}}
  \preauthor{\centering\large\emph}
  \postauthor{\par}
  \predate{\centering\large\emph}
  \postdate{\par}
  \date{Thursday 2 May 2019}

\usepackage{float}
\usepackage{graphicx}
\usepackage{subfig}
\usepackage{fancyhdr}
\pagestyle{fancy}
\usepackage{truncate}
\renewcommand{\subsectionmark}[1]{\markright{#1}{}}
\fancyhf{}
\lhead{\small\truncate{400pt}{\rightmark}}
\rhead{\small\hyperref[toc]{Table Of Contents}}
\rfoot{Page \thepage}
\usepackage{caption}
\usepackage{listings}
\usepackage{attachfile}
\makeatletter\renewcommand*{\fps@figure}{H}\makeatother
\usepackage{cleveref}

\usepackage{xcolor}
\definecolor{block-gray}{gray}{0.85}
\usepackage{environ}
\NewEnviron{quoteblock}
{\colorbox{block-gray}{
\parbox{\dimexpr\linewidth-2\fboxsep\relax}{
\small\addtolength{\leftskip}{10mm}
\addtolength{\rightskip}{10mm}
\BODY}}
}
\renewcommand{\quote}{\quoteblock}
\renewcommand{\endquote}{\endquoteblock}
\ifdef{\printbibliography}{
   \defbibheading{subsubbibliography}[\refname]{\subsubsection*{#1}}
   \let\oldprintbibliography\printbibliography
   \renewcommand{\printbibliography}[1]{
      \phantomsection
      \addcontentsline{toc}{section}{References}
      \oldprintbibliography[title={References},heading=subsubbibliography]
      }
}{

}

\begin{document}
\maketitle

\label{toc}

\hypertarget{a-method-of-parking-space-detection-based-on-image-segmentation-and-lbp}{%
\subsection{A method of Parking space detection based on image segmentation and LBP}\label{a-method-of-parking-space-detection-based-on-image-segmentation-and-lbp}}

\hypertarget{abstract}{%
\subsubsection{Abstract}\label{abstract}}

The article was written by \autocite{Lixia_2012} and cited \href{https://scholar.google.com/scholar?cluster=2180601932752364489\&hl=en\&as_sdt=2005\&sciodt=0,5}{16} times according to Google Scholar. It presents a method for detecting occupancy level using a combination of mean shift areas weighted count and LBP code histogram classification through SVM and Markov Random Fields.

\hypertarget{hypothesis}{%
\subsubsection{Hypothesis}\label{hypothesis}}

Using mean shift segmentation and then LBP code histogram classification using a SVM for texture detection can achieve more than 97\% occupancy level detection rate.

\hypertarget{evidence-and-results}{%
\subsubsection{Evidence and Results}\label{evidence-and-results}}

1439 parking spot images were used for training the LBP-based SVM classifier. Then, 1225 test images. Results are presented in a table containing detection rate (TPR), missing rate (FNR) and virtual rate. It is not clear from context what do \emph{virtual rate} means.

\hypertarget{contribution}{%
\subsubsection{Contribution}\label{contribution}}

The most important contribution of this paper is that it demonstrate how to learn a threshold of a number of segments using mean shift segmentation.

A second contribution is the use of LBP histograms to generate a feature code that can be used as input into other classifiers, in this case an SVM.

\hypertarget{weaknesses}{%
\subsubsection{Weaknesses}\label{weaknesses}}

This paper is badly written. It depends on pre annotated rectangular areas. These areas are then classified into vacant or occupied.

For the first part, the method must learn a threshold g for each parking space and lighting conditions it encounters. Similar gray levels are assumed between all the parking spots in the camera view.

Finally, the authors mention that occlusions by temporary objects like other vehicles on the traffic lane harm the detection accuracy.

\hypertarget{future-work}{%
\subsubsection{Future Work}\label{future-work}}

The authors propose using vehicle tracking in parking spots to overcome occlusions.

\printbibliography


\end{document}
